%** language french/english should be passed to \documentclass
\documentclass[a4paper, french, 10pt]{article}

%** title, author and date (classic latex)
\title{Un super beau titre}
\author{Mews Labs}
\date{\today}

%** options are :
%** [kpfonts, lmodern, libertine, newtx] to chose the font
%** [draft] add a draft watermark
\usepackage[lmodern]{ml_article}

%** header and foot content
\hdcontent{Du contenu ici}
\ftcontent{Andesite Solutions \&{} Mews Labs}

%** override document main color
\colorlet{mmain}{lineD}

%** set dir path for figures
\graphicspath{{figures/}{logo/}}

%** choose logos
\firstlogo{mewslabs}
\secondlogo{andesite}


%%%%%%%%%%%%%%%%%%%%%%%%%%%%%%%%%%%%%%%%%%%%%%%%%%
%** user own def

\newcommand{\W}{\Omega}
\newcommand{\w}{\omega}
\newcommand{\wz}{\omega_0}
\newcommand{\wn}{\omega_n}

\newcommand\st{\text{St}}
\newcommand\CD{C_{\text{D}}}
\newcommand\CLz{\tilde{C}_{\text{L}}}

\newcommand{\Ed}{E_{\text{diss}}}
\newcommand{\Ea}{E_{\text{app}}}

\newcommand{\ymax}{y_{\text{max}}}
\newcommand{\cmax}{\chi_{\text{max}}}

\begin{document}
%%%%%%%%%%%%%%%%%%%%%%%%%%%%%%%%%%%%%%%%%%%%%%%%%%
%%%%%%%%%%%%%%%%%%%%%%%%%%%%%%%%%%%%%%%%%%%%%%%%%%
%%%%%%%%%%%%%%%%%%%%%%%%%%%%%%%%%%%%%%%%%%%%%%%%%%

\maketitle
\abstract{\lipsum[1]}
%% \tableofcontents



%%%%%%%%%%%%%%%%%%%%%%%%%%%%%%%%%%%%%%%%%%%%%%%%%%
%%%%%%%%%%%%%%%%%%%%%%%%%%%%%%%%%%%%%%%%%%%%%%%%%%
%%%%%%%%%%%%%%%%%%%%%%%%%%%%%%%%%%%%%%%%%%%%%%%%%%
\section{Avec \LaTeX{} on peut écrire des maths}

On s'intéresse aux vibrations d'une corde de longueur $L$, tendue à la tension $H$ et de masse linéique $m$
soumise à une excitation périodique en temps
%
\begin{equation}
    \label{eq:wave}
    m\ddpart{y}{t} \,+\, 2m\wz\zeta\dpart{y}{t} \,-\, H \ddpart{y}{x} \;=\; F(x)\cos(\w t) \;,
\end{equation}
%
avec les conditions aux limites suivantes:
%
\begin{mitemize}
    \item la corde est fixée en ses extrémités: $y(0, t) = y(L, t) = 0$;
    \item les conditions initiales sont: $y(x, 0) = y^0(x)$ et $\dpart{y}{t}(x, 0) = \dot{y}^0(x)$.
\end{mitemize}
%
La pulsation propre de la corde est $\wz=\frac{\pi}{L}\sqrt{\frac{H}{m}}$, et on suppose que $\zeta < 1$.
En adimensionnant $y$ par $d$, le diamètre de la corde, $x$ par $L$ et $t$ par $\sfrac{1}{\wz}$, \equref{wave}
devient
%
\begin{equation}
    \label{eq:wadim}
    \ddpart{y}{t} \,+\, 2\zeta\dpart{y}{t} \,-\, \frac{1}{\pi^2} \ddpart{y}{x} \;=\; A(x)\cos(\Omega t) \;,
\end{equation}
%
où $\Omega = \sfrac{\w}{\wz}$ et $A(x) = \sfrac{F(x)}{(m\wz^2 d)}$. La solution générale à ce problème est
données en \secref{fullres}. Dans la suite, on s'intéressera aux cas particuliers où $F(x)$ est une constante
(c'est à dire une excitation uniforme en espace) d'une part, et dans le cas d'une excitation locale d'autre part.
On se placera également dans le régime permanent.

%%%%%%%%%%%%%%%%%%%%%%%%%%%%%%%%%%%%%%%%%%%%%%%%%%
%%%%%%%%%%%%%%%%%%%%%%%%%%%%%%%%%%%%%%%%%%%%%%%%%%
\subsection{Une sous-section}

\lipsum[1]

%%%%%%%%%%%%%%%%%%%%%%%%%%%%%%%%%%%%%%%%%%%%%%%%%%
%%%%%%%%%%%%%%%%%%%%%%%%%%%%%%%%%%%%%%%%%%%%%%%%%%
\subsection{Une autre sous-section}

\lipsum[3]




%%%%%%%%%%%%%%%%%%%%%%%%%%%%%%%%%%%%%%%%%%%%%%%%%%
%%%%%%%%%%%%%%%%%%%%%%%%%%%%%%%%%%%%%%%%%%%%%%%%%%
%%%%%%%%%%%%%%%%%%%%%%%%%%%%%%%%%%%%%%%%%%%%%%%%%%
\section{Une autre section}

\lipsum[4-6]

\begin{mfig}
  \includegraphics[width=0.67\textwidth]{mewslabs.png}
\mcaption{Gros titre important}{La légende en détails.}
\end{mfig}




\clearpage
\appendix
%%%%%%%%%%%%%%%%%%%%%%%%%%%%%%%%%%%%%%%%%%%%%%%%%%
%%%%%%%%%%%%%%%%%%%%%%%%%%%%%%%%%%%%%%%%%%%%%%%%%%
%%%%%%%%%%%%%%%%%%%%%%%%%%%%%%%%%%%%%%%%%%%%%%%%%%
\section{Les annexes}

\lipsum[5]


%%%%%%%%%%%%%%%%%%%%%%%%%%%%%%%%%%%%%%%%%%%%%%%%%%
%%%%%%%%%%%%%%%%%%%%%%%%%%%%%%%%%%%%%%%%%%%%%%%%%%
%%%%%%%%%%%%%%%%%%%%%%%%%%%%%%%%%%%%%%%%%%%%%%%%%%
\end{document}
